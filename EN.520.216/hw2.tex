\documentclass{article}
\usepackage[utf8]{inputenc}

\textwidth=6.5in
\textheight=8.5in
\voffset=-54pt
\oddsidemargin=0in
\evensidemargin=0in
\setlength{\parskip}{8pt}
\setlength{\parindent}{0pt}
\newcommand{\tc}{\textcolor{blue}}
\usepackage[normalem]{ulem}
\usepackage[bottom]{footmisc}
\usepackage{tikz-cd}
\usepackage{amsmath, amssymb, amsthm}
\usepackage{ifthen}
\usepackage[inline]{enumitem}
\setlist{topsep=0pt}
\usepackage{xcolor}
\usepackage{parskip}
\usepackage{blindtext}
\usepackage{multicol}
%operator names
\DeclareMathOperator{\coker}{coker}
\DeclareMathOperator{\im}{im}
\DeclareMathOperator{\coord}{coord}
\DeclareMathOperator{\id}{id}
\DeclareMathOperator{\ob}{\mathrm{ob}}
\DeclareMathOperator{\mor}{\mathrm{mor}}
\DeclareMathOperator*{\colim}{\mathrm{colim}}
\newcommand{\op}{\mathrm{op}}
\newcommand{\co}{\mathrm{co}}
\newcommand{\Nat}{\mathrm{Nat}}
\newcommand{\Hom}{\mathrm{Hom}}
\newcommand{\Map}{\mathrm{Map}}
\newcommand{\End}{\mathrm{End}}
\newcommand{\Aut}{\mathrm{Aut}}
\newcommand{\Sym}{\mathrm{Sym}}
\newcommand{\ev}{\mathrm{ev}}
\newcommand{\Stab}{\mathrm{Stab}}
%blackboard letters
\newcommand{\FF}{\mathbb{F}}
\newcommand{\CC}{\mathbb{C}}
\newcommand{\QQ}{\mathbb{Q}}
\newcommand{\RR}{\mathbb{R}}
\newcommand{\ZZ}{\mathbb{Z}}
\newcommand{\NN}{\mathbb{N}}
\newcommand{\kk}{\mathbbe{k}}
%categories
\newcommand{\A}{\mathcal{A}}
\newcommand{\B}{\mathcal{B}}
\newcommand{\C}{\mathcal{C}}
\newcommand{\D}{\mathcal{D}}
\newcommand{\set}{\mathsf{Set}}
\newcommand{\grp}{\mathsf{Grp}}
\newcommand{\ab}{\mathsf{Ab}}
%theorems, defns, etc
\newtheorem{thm}{Theorem}[section]
\theoremstyle{definition}
\newtheorem*{defn*}{Definition}
\newtheorem*{thm*}{Theorem}
%This makes the problem environment work.
\theoremstyle{definition}
\newtheorem{problem}{Problem}
%This makes the solution environment work.
\makeatletter\newenvironment{solution}[1][Solution]{\par\pushQED{\qed}%
\normalfont 
\topsep6\p@\@plus6\p@\relax\trivlist\item\relax{\itshape#1\@addpunct{.}}\hspace\labelsep\ignorespaces}{%
\popQED\endtrivlist\@endpefalse}
\makeatother


\title{EN.520.216 Homework 2}
\author{LJ Gonzales}
\date{February 2023}

\begin{document}
\maketitle

\begin{prob}
	\begin{enumerate}
		\item First notice that the voltage at the anode of D4 is 0.6V (0V ground+dropoff voltage), and likewise the voltages at the anodes of D1, D2, and D3 are 1.2V, 0.9V, and 0.9V respectively. This is when we use the first-order model where a diode is considered to be a source with the positive terminal on the side opposing current flow.
		Because the supply voltage is maintained at 18V, it follows that the respective currents are $I_1=\frac{18-1.2}{2200}= 7.63mA$, $I_2=\frac{18-0.9}{2700}= 6.33mA$, and $I_3=\frac{18-0.9}{1000}= 17.1mA$.
		By Kirchoff's law the current through the source is then 31.06mA.
	
	\item All of the diodes are forward-biased, since the voltage at the anode is lower than the voltage at the cathode.

	\item The power absorbed by D4 can be computed from $P=IV$ where I and V are the current and voltage through the component. Using $I_d$ and 0.6V we get $P= 18.636mW$.
	\end{enumerate}
\end{prob}

\begin{prob}
	\begin{enumerate}
	\item
		\begin{tikzpicture}
\draw[gray] plot [smooth] coordinates {(0,1) (2,1) (3,0) (6,0)};
\draw[gray] plot [smooth] coordinates {(0,3) (2, 3) (3,2) (6,2)};
\draw[gray, dashed]	(0,1.5) -- (6,1.5);
\node[right] at (6,2) {$E_{c,n}$};
\node[right] at (6,1.5) {$E_{f}$};
\node[right] at (6,0) {$E_{v,n}$};
\node[left] at (0,3) {$E_{c,n}$};
\node[left] at (0,1) {$E_{v,p}$};
\node[right] at (0,-1) {P region};
\node[left] at (6,-1) {N region};
	\end{tikzpicture}
\item 
\item The contact potential is given by $\Phi \ln{\frac{N_AN_D}{n_i^2}}$ where $\Phi =\frac{kT}{q} \approx 26mV$ at 300K.
	
\end{enumerate}
\end{prob}
\begin{figure}[h]
	\includegraphics[width =\textwidth]{graph2.png}
	\label{grappph}
	\caption{Temperature vs. built-in voltage}
\end{figure}

\begin{prob}
	\begin{enumerate}
		\item The diode is not always forward biased: for non-zero capacitance, charge accumulates on the positive end of the capacitor which seeks to return through the diode when the voltage source goes negative.
		\item The diode makes sure that only current in the positive direction goes through the load.
			The capacitor makes sure that the voltage source fails to pull any current in the opposite direction.

	\item We first note that the saturation current of this diode vastly diverges from the one we would expect from a silicon diode so we need to compute its voltage drop before applying the constant drop model.
		The circuit can be partitioned in two periods of operation: a) while the diode is in forward conduction (mostly on the rising of the voltage source) and b) when the diode is reversed biased. \\
		The latter happens not only when the voltage source is in the negative region but also right after it reaches its peak, since the capacitor would have already been charged and discharges through the resistor more slowly than the voltage on the voltage source decreases back down (otherwise, there would be no ripple, just a half-rectified wave). \\
		Hence, the time period that the capacitor discharges through the load resistor (with no help from the voltage source) is [$\frac{1}{4}T, t_1]$, where T is the period of the sinusoid and $t_1$ is the timestamp at which the diode is back in forward conduction. \\
		We know $t_1$ to satisfy  $\sin{20\pi t_1}=V_{DOn}$, and if we limit ourselves to its  first instance we have  $t_1=\frac{1}{20\pi}\sin ^{-1}(V_{DOn})+\frac{1}{10}$. (adding the length of one period).
		We want to make sure that during that known amount of time, the voltage across the load does not decrease more than $1-V_{DOn}-0.5$, from its original value of $1-V_{DOn}$.
		Within this time period, the load is just looking at a capacitor, and we know the equation of this setup to be
		\begin{align*}
			\int _{t=\frac{1}{4}T}^{\frac{1}{20\pi}\sin ^{-1}(V_{DOn})+\frac{1}{10}}\frac{1}{RC}dt = \int _{1-V_{DOn}-0.5}^{1-V_{DOn}}\frac{1}{V_c(t)}dV_c(t) \\
			\therefore R= \frac{\frac{1}{20\pi}\sin ^{-1}(V_{DOn})+ \frac{3}{40}}{10\times10^{-6}\ln \big(\frac{1-V_{DOn}}{1-V_{DOn}-0.5}\big)}
		\end{align*}
		Just as we did to obtain the 0.7V estimation of the silicon diode, we can set up a voltage source, diode, and resistor in series and solve under the exponential model to get $V_{DOn}$. With a 1V source and 10k$\Omega$ resistor, we have
		$1-V_{DOn}-1.8\times10^{-2}(e^{V_{DOn}/\Phi_T}-1)=0$, which gives a $V_{DOn}$ value of 0.103V. Choosing other values of V and R as well neighbors this value.
		Putting this back in the equation, we get $R\approx9402\Omega$, our final answer. \\
	\item The maximum output voltage under the constant drop model is given by the max source voltage minus one voltage drop, or about 0.897V
	
	\item
	\end{enumerate}
\end{prob}

\begin{prob}
	
\end{prob}
\end{document}
