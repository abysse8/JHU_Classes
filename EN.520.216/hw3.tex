\documentclass{article}
\usepackage[utf8]{inputenc}
\textwidth=6.5in
\textheight=8.5in
\voffset=-54pt
\oddsidemargin=0in
\evensidemargin=0in
\setlength{\parskip}{8pt}
\setlength{\parindent}{0pt}
\newcommand{\tc}{\textcolor{blue}}
\usepackage[normalem]{ulem}
\usepackage[bottom]{footmisc}
\usepackage{tikz-cd}
\usepackage{amsmath, amssymb, amsthm}
\usepackage{ifthen}
\usepackage[inline]{enumitem}
\setlist{topsep=0pt}
\usepackage{xcolor}
\usepackage{parskip}
\usepackage{blindtext}
\usepackage{multicol}
%operator names
\DeclareMathOperator{\coker}{coker}
\DeclareMathOperator{\im}{im}
\DeclareMathOperator{\coord}{coord}
\DeclareMathOperator{\id}{id}
\DeclareMathOperator{\ob}{\mathrm{ob}}
\DeclareMathOperator{\mor}{\mathrm{mor}}
\DeclareMathOperator*{\colim}{\mathrm{colim}}
\newcommand{\op}{\mathrm{op}}
\newcommand{\co}{\mathrm{co}}
\newcommand{\Nat}{\mathrm{Nat}}
\newcommand{\Hom}{\mathrm{Hom}}
\newcommand{\Map}{\mathrm{Map}}
\newcommand{\End}{\mathrm{End}}
\newcommand{\Aut}{\mathrm{Aut}}
\newcommand{\Sym}{\mathrm{Sym}}
\newcommand{\ev}{\mathrm{ev}}
\newcommand{\Stab}{\mathrm{Stab}}
%blackboard letters
\newcommand{\FF}{\mathbb{F}}
\newcommand{\CC}{\mathbb{C}}
\newcommand{\QQ}{\mathbb{Q}}
\newcommand{\RR}{\mathbb{R}}
\newcommand{\ZZ}{\mathbb{Z}}
\newcommand{\NN}{\mathbb{N}}
\newcommand{\kk}{\mathbbe{k}}
%categories
\newcommand{\A}{\mathcal{A}}
\newcommand{\B}{\mathcal{B}}
\newcommand{\C}{\mathcal{C}}
\newcommand{\D}{\mathcal{D}}
\newcommand{\set}{\mathsf{Set}}
\newcommand{\grp}{\mathsf{Grp}}
\newcommand{\ab}{\mathsf{Ab}}
%theorems, defns, etc
\newtheorem{thm}{Theorem}[section]
\theoremstyle{definition}
\newtheorem*{defn*}{Definition}
\newtheorem*{thm*}{Theorem}
%This makes the problem environment work.
\theoremstyle{definition}
\newtheorem{problem}{Problem}
%This makes the solution environment work.
\makeatletter\newenvironment{solution}[1][Solution]{\par\pushQED{\qed}%
\normalfont 
\topsep6\p@\@plus6\p@\relax\trivlist\item\relax{\itshape#1\@addpunct{.}}\hspace\labelsep\ignorespaces}{%
\popQED\endtrivlist\@endpefalse}
\makeatother

\title{EN.520.216 Homework 3}
\author{LJ Gonzales}
\date{February 2023}

\begin{document}
\maketitle
\begin{prob}
	
\end{prob}
\begin{prob}
	We use the fact that the current is expressed by $I_D=\frac{k'W}{2L}((V_{GS}-V_T)V_{DS}-V_{DS}^2)$ in saturation, and see if the resulting $V_O$ makes sense, and if necessary, evaluate the linear mode equation (since $V_{GS}-V_T > 0 \forall$ R).
\end{prob}

\begin{prob}
\begin{enumerate}
\item	$C_1$ removes any potential bias from the ac voltage source that might interfere with the quiescent point of operation.
	$C_2$ also removes bias at the output such that the only output is the ac amplified signal.

\item At the dc operating point, the gate sees the output of a voltage divider, with value $V_{dd}\frac{R_g}{R_g+R_b}= 0.3V=V_{GS}$.
	This gives a drain current $I_D=\frac{k'W}{2L}(V_{GS}-V_T)=3.072\mu A$ (we are told that the transistor is in saturation mode). 
\item $\lambda=0$, therefore it follows that  $\frac{dI_D}{dV_{DS}}=0$ and $r_o$ is reduced to an open circuit.
	For small signals, we can ground all commons and treat capacitors as wires, as in figure %\ref{smallsignaldraw}.
	The voltage at the gate is the output of a voltage divider formed by $R_1$ and $R_2||R_G||R_B$, where the former is $2k\Omega$ and the latter $\approx1.66k\Omega$.
	We write this as $V_{GS}=V_i\frac{1.66}{3.66}$.
	The transconductance parameter $g_m$ can be derived to be $2\sqrt{K_nI_D}$ where $K_n=\frac{k'W}{2L}=0.48mAV^{-2}$ and $I_d$ is as before, so $g_m=0.0768mAV^{-1}$.
	The current through $R_D$ is then given by $0.0768v_{gs}$, and $V_o= 46.08v_{gs}$.
	$\frac{V_o}{V_i}=\frac{v_{gs}46.08}{V_{gs}3.66/1.66} = 20.90$, which is our gain.
	\end{enumerate}
\end{prob}
\end{document}
