\documentclass{article}
\usepackage[utf8]{inputenc}
\usepackage{amsthm}
\usepackage{amsmath}
\newtheoremstyle{mystyle}% name
  {\topsep}% Space above
  {\topsep}% Space below
  {\normalfont}% Body font
  {}% Indent amount
  {\bfseries}% Theorem head font
  {}%Punctuation after theorem head
  {.5em}%Space after theorem head
  {}% theorem head spec
\theoremstyle{mystyle}
\newtheorem{prob}{Problem}
\usepackage{graphicx}
\usepackage{wrapfig}

\title{EN.520.216 Homework 3}
\author{LJ Gonzales}
\date{February 2023}

\begin{document}
\maketitle
\begin{prob}
	
\end{prob}
\begin{prob}
	We use the fact that the current is expressed by $I_D=\frac{k'W}{2L}((V_{GS}-V_T)V_{DS}-V_{DS}^2)$ in saturation, and see if the resulting $V_O$ makes sense, and if necessary, evaluate the linear mode equation (since $V_{GS}-V_T > 0 \forall$ R).
\end{prob}

\begin{prob}
\begin{enumerate}
\item	$C_1$ removes any potential bias from the ac voltage source that might interfere with the quiescent point of operation.
	$C_2$ also removes bias at the output such that the only output is the ac amplified signal.

\item At the dc operating point, the gate sees the output of a voltage divider, with value $V_{dd}\frac{R_g}{R_g+R_b}= 0.3V=V_{GS}$.
	This gives a drain current $I_D=\frac{k'W}{2L}(V_{GS}-V_T)=3.072\mu A$ (we are told that the transistor is in saturation mode). 
\item $\lambda=0$, therefore it follows that  $\frac{dI_D}{dV_{DS}}=0$ and $r_o$ is reduced to an open circuit.
	For small signals, we can ground all commons and treat capacitors as wires, as in figure %\ref{smallsignaldraw}.
	The voltage at the gate is the output of a voltage divider formed by $R_1$ and $R_2||R_G||R_B$, where the former is $2k\Omega$ and the latter $\approx1.66k\Omega$.
	We write this as $V_{GS}=V_i\frac{1.66}{3.66}$.
	The transconductance parameter $g_m$ can be derived to be $2\sqrt{K_nI_D}$ where $K_n=\frac{k'W}{2L}=0.48mAV^{-2}$ and $I_d$ is as before, so $g_m=0.0768mAV^{-1}$.
	The current through $R_D$ is then given by $0.0768v_{gs}$, and $V_o= 46.08v_{gs}$.
	$\frac{V_o}{V_i}=\frac{v_{gs}46.08}{V_{gs}3.66/1.66} = 20.90$, which is our gain.
	\end{enumerate}
\end{prob}
\end{document}
