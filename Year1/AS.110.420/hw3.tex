\documentclass{article}
\usepackage[utf8]{inputenc}

\title{AS.110.420 HW3}
\author{LJ Gonzales, Jed's section}
\date{February 2023}
\textwidth=6.5in
\textheight=8.5in
\voffset=-54pt
\oddsidemargin=0in
\evensidemargin=0in
\setlength{\parskip}{8pt}
\setlength{\parindent}{0pt}
\newcommand{\tc}{\textcolor{blue}}
\usepackage[normalem]{ulem}
\usepackage[bottom]{footmisc}
\usepackage{tikz-cd}
\usepackage{amsmath, amssymb, amsthm}
\usepackage{ifthen}
\usepackage[inline]{enumitem}
\setlist{topsep=0pt}
\usepackage{xcolor}
\usepackage{parskip}
\usepackage{blindtext}
\usepackage{multicol}
%operator names
\DeclareMathOperator{\coker}{coker}
\DeclareMathOperator{\im}{im}
\DeclareMathOperator{\coord}{coord}
\DeclareMathOperator{\id}{id}
\DeclareMathOperator{\ob}{\mathrm{ob}}
\DeclareMathOperator{\mor}{\mathrm{mor}}
\DeclareMathOperator*{\colim}{\mathrm{colim}}
\newcommand{\op}{\mathrm{op}}
\newcommand{\co}{\mathrm{co}}
\newcommand{\Nat}{\mathrm{Nat}}
\newcommand{\Hom}{\mathrm{Hom}}
\newcommand{\Map}{\mathrm{Map}}
\newcommand{\End}{\mathrm{End}}
\newcommand{\Aut}{\mathrm{Aut}}
\newcommand{\Sym}{\mathrm{Sym}}
\newcommand{\ev}{\mathrm{ev}}
\newcommand{\Stab}{\mathrm{Stab}}
%blackboard letters
\newcommand{\FF}{\mathbb{F}}
\newcommand{\CC}{\mathbb{C}}
\newcommand{\QQ}{\mathbb{Q}}
\newcommand{\RR}{\mathbb{R}}
\newcommand{\ZZ}{\mathbb{Z}}
\newcommand{\NN}{\mathbb{N}}
\newcommand{\kk}{\mathbbe{k}}
%categories
\newcommand{\A}{\mathcal{A}}
\newcommand{\B}{\mathcal{B}}
\newcommand{\C}{\mathcal{C}}
\newcommand{\D}{\mathcal{D}}
\newcommand{\set}{\mathsf{Set}}
\newcommand{\grp}{\mathsf{Grp}}
\newcommand{\ab}{\mathsf{Ab}}
%theorems, defns, etc
\newtheorem{thm}{Theorem}[section]
\theoremstyle{definition}
\newtheorem*{defn*}{Definition}
\newtheorem*{thm*}{Theorem}
%This makes the problem environment work.
\theoremstyle{definition}
\newtheorem{problem}{Problem}
%This makes the solution environment work.
\makeatletter\newenvironment{solution}[1][Solution]{\par\pushQED{\qed}%
\normalfont 
\topsep6\p@\@plus6\p@\relax\trivlist\item\relax{\itshape#1\@addpunct{.}}\hspace\labelsep\ignorespaces}{%
\popQED\endtrivlist\@endpefalse}
\makeatother

\begin{document}
\maketitle

\begin{prob}
	%Problem 1
	We note that $B$ and $B^c$ are mutually exclusive events. Because $(A \cap B) \subseteq B$ and  $(A \cap B^c) \subseteq B^c$, $AB$ and $AB^c$ are also mutually exclusive events.
	By the axiom of subbadditivity, it then holds that $P((A \cap B) \cup (A \cap B^c)) = P(AB^c)+P(AB)$. We can equivalently write the left hand side of this statement as $P(A \cup (B \cap B^c))$.
	However, because $B$ and $B^c$ are mutually exlusive, their interesection is the null set, and any set $\cup$ the null set is just that set itself.
	It follows that $P(A)=P(A-B)+P(AB)$, writing the previous statement in a different way. Rearranging, we have the result. 
\end{prob}

\begin{prob}
	%Problem 2
\begin{enumerate}
	\item Let D be the event that the market goes down, and T the event that trading exceeds 1 billion shares, such that $P(D)=0.05, P(T)=0.2, P(D \cap T)=0.03$.
		We want to compute $P(D \cap T^c)$. However, $T$ and $T^c$ form an exhaustive and mutually exclusive partition of the set, such that $P(B)=P(B \cap T)+P(B \cap T^c)$.
		By simple interpolation of known variables, it follows that $P(D \cap T^c)=0.02$.
	
	\item We begin by assigning events to their conditional probability formula. We define E to be the event that a person exercises regularly and U the event that they are under 20 years old.
	We have $P(E|U)=\frac{P(E \cap U)}{P(U)}=0.75$, $P(U|E)=\frac{P(E \cap U)}{P(E)}=0.6$. Because we also know that $P(E \cap U)=0.1$, we can deduce that $P(U)=\frac{0.1}{0.75}$, and $P(E)=\frac{0.1}{0.6}$.
	We want to find $P(E|U')=\frac{P(E \cap U')}{P(U')}$.
	Notice, however, that U and its complement forms an exhaustive and mutually exclusive partition of the sample space, such that $P(E)=P(E \cap U)+P(E \cap U')$, where both $P(E)$ and $P(E \cap U)$ are known.
	Hence the numerator of the desired answer is  $\frac{0.1}{0.6}-0.1$, and the denominator is $1-P(U)=1-\frac{0.1}{0.75}$, giving us a final answer of 0.0769.

	\item Letting R and W represent the event of rain and wind, We know that $P(R)=0.5, P(W)=0.3, P(R^c \cap W^c)=0.1$. 
		By De Morgan's law the latter is equal to $P((R \cup W)^c)=1-P(R \cup W)$. It would then follow that $P(R \cup W)=0.9$.
		However, this cannot be correct, since the union of two sets is at most equal to their sums, with equality only in the case that the events are disjunct. The metereologist is a fraud.
\end{enumerate}	
\end{prob}

\begin{prob}
	%Problem 3
	Our sample space $\Omega$ is the set  $\{H_1H_2, H_1T_2, T_1H_2, T_1T_2\}$, and we seek $P(H_1|(H_1 \cup H_2)=\frac{P(H_1 \cap (H_1 \cup H_2))}{P((H_1 \cup H_2))}$.
	Note that $H_1 \subset (H_1 \cup H_2)$ so the numerator is just 1/2. The denominator can be obtained by the inclusion-exclusion principle as $P(H_1)+P(H_2)-P(H_1H_2)=\frac{1}{2}+\frac{1}{2}-\frac{1}{4}=\frac{3}{4}$. The answer is $\frac{1}{2}\times\frac{4}{3}=\frac{2}{3}$.
\end{prob}

\begin{prob}
	%Problem 4
	\begin{enumerate}
		\item For the next 3 questions we let R be the event that the second card drawn is the same rank as the first, and C the event that they are the same color.
		We want to compute $P(C^c|R)=\frac{P(C^c \cap R)}{P(R)}$. We can write the numerator as $\frac{2}{51}$, as there are 2 cards left of the same rank and different color that could be chosen.
		The denominator is  $\frac{3}{51}$, as there are 3 cards left of the same suit (four originally, one having been removed). This gives us a final answe of $\frac{2}{3}$, coherent with our expectation.

	\item $P(R|C^c)=\frac{P(R \cap C^c)}{P(C^c)}$, where, by a similar argument as the previous, the numerator is $\frac{2}{51}$ and the denominator $\frac{26}{51}$, for a final answer of $\frac{1}{13}$.
		

	\item $P(R|C)=\frac{P(R \cap C)}{P(C)}=\frac{\frac{1}{51}}{\frac{25}{51}} =\frac{1}{25}$.
	\end{enumerate}
\end{prob}

\begin{prob}	
	%Problem 5 dealt 6 cards prob diam+heart
	We know that $P(D \cup H)=P(D)+P(H)-P(DH)$, where we want to find $P(DH)$. Note that $P(D \cup H)=1-(P(D^cH^c))$ by the complements rule and De Morgan's law, where the latter is easily computed by $\frac{(26)_6}{(52)_6}$ (if we exclude all diamonds and hearts, we're picking a hand from half the deck).
	Likewise, we know that the probability of getting a diamond on any given card is $\frac{1}{4}$, so the probability at least 1 diamond is $1-P(\text{no diamond})=1-(\frac{3}{4})^6$, and likewise for P(H).
	It follows that our desired probability is $2\times(1-(\frac{3}{4})^6)-(1-\frac{230230}{20358520})$, which evaluates to about 0.655.
\end{prob}

\begin{prob}
	%Problem 6 You roll a 6-sided die 7 times. The event A is you roll a 4, 5 or 6; B is you roll an odd number. Compute P (B|A)
	Note that $P(B|A)=\frac{P(B \cap A)}{P(A)}$. The intersection is is clearly $\frac{1}{3}P(A)$, since only one out of the three outcomes of A also qualify for B. The answer is $\frac{1}{3}$.
\end{prob}

\begin{prob}
	%Problem 7 chance that 30-long ternary has exactly 10 of same digit
	\begin{enumerate}
		\item We consider the $3^{30}$ equally probable outcomes of the 30-long sequence, and given any of 3 digits that must appear 10 times, there are ${30 \choose 10}$ ways to rearrange them inside the digit, and a further $2^{20}$ other digits to fill out the rest of the number.
		The probability of this happening is $\frac{3\times{30 \choose 10}\times2^{20}}{3^{30}}=\frac{3500497960960}{7625597484987}\approx 0.4590$.

	\item Given the event O=a 1 occurred exactly 10 times, we want to compute $P(O|A)=\frac{P(O \cap A)}{P(A)}$, where the denominator is the part computed in a) and the numerator is $\frac{1}{3}P(A)$, since we choose 1 number to occur over the possible 3.
		The answer is then $1/3$.
	\end{enumerate}
\end{prob}

\begin{prob}
	%Problem 8
	\begin{enumerate}
		\item By direct application of the exchangeability property, $P(M_i \cap M_j^c)$ is equal to $P(M_1)P(M_2^c)$ with n objects or $\frac{1}{n}\frac{n-2}{n-1}$. Note the intentional placement of the 'matched' with the 'failed matched' event in $M_1M_2^c$. Had they been in the reverse order, independence could not be assumed, since there is a chance that M1 could have failed with M2's number.

		\item By the conditional property formula, this is $\frac{P(M_i \cap M_j^c)}{P(M_j^c}=\frac{\frac{1}{n}\frac{n-1}{n-2}}{\frac{n-1}{n}}$ or simply $\frac{1}{n-2}$.
	\end{enumerate}
\end{prob}

\begin{prob}
	%Problem 9
	The outcome of the coin flip is mutually exclusive and exhaustive for the sample space, so if we let R represent the event that we roll a 6,
	we can write $P(R)=P(R \cap H)+P(R \cap T)$, where $P(R \cap H)=P(H)\times P(R|H)=\frac{1}{2}\times\frac{1}{6}$, and $P(R \cap T)=P(T)\times P(R|T)=\frac{1}{2}\times\frac{5}{36}$.
	Hence the probability of rolling a 6 is given about 0.153 all things considered.
\end{prob}

\begin{prob}
	%Problem 10 urn
	There are only two possible outcomes from the first step of the experiment: either two blue balls are retrieved from one of the urns or one blue and one red, and this is done twice. Call the first event B and the second A, for a total set $\{AA, AB, BB\}$.
	We can then write the event of drawing a blue ball from urn 3, called event W, as $P(W)=P(W \cap AA)+P(W \cap AB)+P(W \cap BB)$, since the set forms a mutually exclusive and exhaustive partition.
	From the conditional probability definition, $P(W)$ is then equal to $P(AA)P(W|AA)+P(AB)P(W|AB)+P(BB)P(W|BB)$. We can read them from the problem to get  $(\frac{1}{3})^2\frac{1}{2}+2(\frac{1}{3}\times\frac{2}{3})\frac{3}{4}+(\frac{2}{3})^2\times1$
\end{prob}

\begin{prob}
	%Problem 11
	The probability of seeing doubles before seven is the probability of seeing double on the first row, +not seeing seven on the first roll and seeing double on the second, and so on.
	$P(D)=P(D_1)+P((S_1^c \cap D_1^c)\cap D_2)+P((S_1^c \cap D_1^c) \cap (S_2^c \cap D_2^c) \cap D_3) \dots = P(D_1)+P(S_1^c \cap D_1^c)P(D_2)+P(S_1^c \cap D_1^c)P(S_2^c \cap D_2^c)P(D_3) \dots$by independence (different rolls have nothing to do with each other, but cannot say the same for $S_i^c \cap D_i$).

	Note that the probability of not getting a seven and also not getting a double on the same roll is the same for any given roll i and is equal to the sample space minus ${11,22,33,44,55,66,61,16,52,25,34,43}$ or $\frac{24}{36}=\frac{2}{3}$.
	Hence we have a geometric series with first term $\frac{1}{6}$ and common ratio $\frac{2}{3}$, which gives a probability of 0.5.
	This makes sense since S and D are equally likely in the sample space and mutually exclusive on any given roll.
\end{prob}
\end{document}
