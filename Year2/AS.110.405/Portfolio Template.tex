\documentclass[11 pt]{article}
        % adjust text size, margins
    \textwidth=7in
    \textheight=9.6in
    \voffset=-54pt
    \oddsidemargin=-.25in
    \evensidemargin=0in
    \setlength{\parskip}{8pt}
    \setlength{\parindent}{0pt}

        % change hyperref options
    \PassOptionsToPackage{hyphens}{url}
    \usepackage[bookmarks,colorlinks,linkcolor=blue,citecolor=blue,pdfstartview=FitH,urlcolor=blue]{hyperref}
    \hypersetup{pdfpagemode=UseNone}

    
    \textwidth=6.5in
\textheight=8.5in
\voffset=-54pt
\oddsidemargin=0in
\evensidemargin=0in
\setlength{\parskip}{8pt}
\setlength{\parindent}{0pt}
\newcommand{\tc}{\textcolor{blue}}
\usepackage[normalem]{ulem}
\usepackage[bottom]{footmisc}
\usepackage{tikz-cd}
\usepackage{amsmath, amssymb, amsthm}
\usepackage{ifthen}
\usepackage[inline]{enumitem}
\setlist{topsep=0pt}
\usepackage{xcolor}
\usepackage{parskip}
\usepackage{blindtext}
\usepackage{multicol}
%operator names
\DeclareMathOperator{\coker}{coker}
\DeclareMathOperator{\im}{im}
\DeclareMathOperator{\coord}{coord}
\DeclareMathOperator{\id}{id}
\DeclareMathOperator{\ob}{\mathrm{ob}}
\DeclareMathOperator{\mor}{\mathrm{mor}}
\DeclareMathOperator*{\colim}{\mathrm{colim}}
\newcommand{\op}{\mathrm{op}}
\newcommand{\co}{\mathrm{co}}
\newcommand{\Nat}{\mathrm{Nat}}
\newcommand{\Hom}{\mathrm{Hom}}
\newcommand{\Map}{\mathrm{Map}}
\newcommand{\End}{\mathrm{End}}
\newcommand{\Aut}{\mathrm{Aut}}
\newcommand{\Sym}{\mathrm{Sym}}
\newcommand{\ev}{\mathrm{ev}}
\newcommand{\Stab}{\mathrm{Stab}}
%blackboard letters
\newcommand{\FF}{\mathbb{F}}
\newcommand{\CC}{\mathbb{C}}
\newcommand{\QQ}{\mathbb{Q}}
\newcommand{\RR}{\mathbb{R}}
\newcommand{\ZZ}{\mathbb{Z}}
\newcommand{\NN}{\mathbb{N}}
\newcommand{\kk}{\mathbbe{k}}
%categories
\newcommand{\A}{\mathcal{A}}
\newcommand{\B}{\mathcal{B}}
\newcommand{\C}{\mathcal{C}}
\newcommand{\D}{\mathcal{D}}
\newcommand{\set}{\mathsf{Set}}
\newcommand{\grp}{\mathsf{Grp}}
\newcommand{\ab}{\mathsf{Ab}}
%theorems, defns, etc
\newtheorem{thm}{Theorem}[section]
\theoremstyle{definition}
\newtheorem*{defn*}{Definition}
\newtheorem*{thm*}{Theorem}
%This makes the problem environment work.
\theoremstyle{definition}
\newtheorem{problem}{Problem}
%This makes the solution environment work.
\makeatletter\newenvironment{solution}[1][Solution]{\par\pushQED{\qed}%
\normalfont 
\topsep6\p@\@plus6\p@\relax\trivlist\item\relax{\itshape#1\@addpunct{.}}\hspace\labelsep\ignorespaces}{%
\popQED\endtrivlist\@endpefalse}
\makeatother

    \usepackage{quiver}

\title{Analysis I Portfolio - Fall 2023} %Title of assignment
\author{Example Name} %Put your name here
\date{} %Writing this as an empty command prevents the date from being printed under your name. Omitting will lead to current date printing.

\begin{document}
\maketitle

\section{Cover Letter}

At the end of the semester, you'll be working on a cover letter for your portfolio that's meant as an opportunity for you to synthesize the different mathematical concepts that you'll be covering in your portfolio. This is a great place to put it!

\pagebreak
%start copying here for Portfolio Assignment Template
\section{Portfolio assignment Template}

%Use the "Portfolio assignment Template" section as a template for each individual portfolio assignment. 
\begin{multicols*}{2} %multicols how-to https://www.overleaf.com/learn/latex/Multiple_columns
    \subsection{Background information}
Does your reader know enough to understand the problem at hand? If not, use this section as a place to (BRIEFLY) mention any relevant background concepts or theorems. Remember to explain them at the level of the audience given! 

\subsection{Task}
Use this section to describe what you're expected to do for the portfolio assignment in question. Remember, this is meant to be self-contained, which means that anyone with the appropriate mathematical background should be able to pick up this portfolio and know what's going on. By the end of this section, your reader should have a sense of the problem or idea that you're planning to address.

\subsection{Strategy}
Use this section to briefly outline how you're planning to address the task that you've explained above. If you're proving a specific theorem, this may be a great place to give a brief outline before going into the nitty-gritty details. 

\columnbreak
\subsection{Solution}
This is where you get into everything and completely address all the nuances needed to address the task you outlined. Even though you'll be talking about details here, make sure that the assignment still follows the outline you've set up. I've formatted this so that the detail of the solution is on the right with the other sections quickly and easily available on the left. This allows your reader to reference between sections easily. 

Don't forget that it is encouraged for you to refer back to past results in the portfolio! If you already proved it once, it's perfectly acceptable to say something along the lines of "As we proved in (Assignment/reference here), result here". This will help with brevity and general readibility!
\end{multicols*}
\pagebreak
%stop copying here.

\end{document}