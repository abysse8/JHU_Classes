\documentclass[11 pt]{article}
\textwidth=6.5in
\textheight=8.5in
\voffset=-54pt
\oddsidemargin=0in
\evensidemargin=0in
\setlength{\parskip}{8pt}
\setlength{\parindent}{0pt}
\newcommand{\tc}{\textcolor{blue}}
\usepackage[normalem]{ulem}
\usepackage[bottom]{footmisc}
\usepackage{tikz-cd}
\usepackage{amsmath, amssymb, amsthm}
\usepackage{ifthen}
\usepackage[inline]{enumitem}
\setlist{topsep=0pt}
\usepackage{xcolor}
\usepackage{parskip}
\usepackage{blindtext}
\usepackage{multicol}
%operator names
\DeclareMathOperator{\coker}{coker}
\DeclareMathOperator{\im}{im}
\DeclareMathOperator{\coord}{coord}
\DeclareMathOperator{\id}{id}
\DeclareMathOperator{\ob}{\mathrm{ob}}
\DeclareMathOperator{\mor}{\mathrm{mor}}
\DeclareMathOperator*{\colim}{\mathrm{colim}}
\newcommand{\op}{\mathrm{op}}
\newcommand{\co}{\mathrm{co}}
\newcommand{\Nat}{\mathrm{Nat}}
\newcommand{\Hom}{\mathrm{Hom}}
\newcommand{\Map}{\mathrm{Map}}
\newcommand{\End}{\mathrm{End}}
\newcommand{\Aut}{\mathrm{Aut}}
\newcommand{\Sym}{\mathrm{Sym}}
\newcommand{\ev}{\mathrm{ev}}
\newcommand{\Stab}{\mathrm{Stab}}
%blackboard letters
\newcommand{\FF}{\mathbb{F}}
\newcommand{\CC}{\mathbb{C}}
\newcommand{\QQ}{\mathbb{Q}}
\newcommand{\RR}{\mathbb{R}}
\newcommand{\ZZ}{\mathbb{Z}}
\newcommand{\NN}{\mathbb{N}}
\newcommand{\kk}{\mathbbe{k}}
%categories
\newcommand{\A}{\mathcal{A}}
\newcommand{\B}{\mathcal{B}}
\newcommand{\C}{\mathcal{C}}
\newcommand{\D}{\mathcal{D}}
\newcommand{\set}{\mathsf{Set}}
\newcommand{\grp}{\mathsf{Grp}}
\newcommand{\ab}{\mathsf{Ab}}
%theorems, defns, etc
\newtheorem{thm}{Theorem}[section]
\theoremstyle{definition}
\newtheorem*{defn*}{Definition}
\newtheorem*{thm*}{Theorem}
%This makes the problem environment work.
\theoremstyle{definition}
\newtheorem{problem}{Problem}
%This makes the solution environment work.
\makeatletter\newenvironment{solution}[1][Solution]{\par\pushQED{\qed}%
\normalfont 
\topsep6\p@\@plus6\p@\relax\trivlist\item\relax{\itshape#1\@addpunct{.}}\hspace\labelsep\ignorespaces}{%
\popQED\endtrivlist\@endpefalse}
\makeatother

\usepackage{quiver}

\begin{document}
    Name: Lj Gonzales

    Assignment: HW 12

    Due Date: Friday December 1
    \break
\begin{problem}
\end{problem}
\begin{solution}
\end{solution} 
\pagebreak
\begin{problem}
	Let $I=[0,1]$. Do the following:
	 \begin{itemize}
		 \item Let $f_n(x)=\frac{x^n}{n}$ on I. Show that $\{f_n(x)\}$ converges uniformly to a differentiable function $f$ on $I$ by finding $f$. Then show that $f'(1)\neq \lim_{n\to\infty}f_n'(1)$.
	         \item For $g$ a Riemann integrable (hence bounded) function on I, find $\lim_{n\to\infty}\int_{0}^{1}\frac{g(x)}{n}dx$.
	\end{itemize}
\end{problem}
\begin{solution}
	We claim that $\{f_n(x)\}\to 0$. To see this, see that 
	 \[
	|f_n(x)-0|=|\frac{x^n}{n}|\leq|\frac{1}{n}|
	.\] 
	If $x$ is in  $[0,1]$. Now for any $\epsilon>0$, we can choose any $N\in\mathbb{N}$ such that $N>\frac{1}{\epsilon}$. Then, for all $n>N$, we have
	 \[
	|f_n(x)-f(x)|<\epsilon
	.\] 
	and since $x$ was chosen arbitrarily in $[0,1]$, the convergence is uniform. We then see that $\lim_{n\to\infty}f_n'(x)=x^{n-1}$, which, evaluated at $x=1$, gives a value of 1. This is not equal to $(0)'|_{x=1}=0$, as sought.
\end{solution}
\begin{solution}
	We will try to estimate the expression $|\lim_{n\to\infty}\int_{0}^{1}\frac{g(x)}{n}dx|$. If we can bound this above by 0 (since it is trivially bounded below by 0), we can use a sandwich argument to conclude that $\lim_{n\to\infty}\int_{0}^{1}\frac{g(x)}{n}dx=0$.\\
	Note that $g(x)$ is bounded, in other words, $|g(x)|<M$ for some real $M$. We use the triangle inequality for integrals and theorem 6.2.4, which states that  $\lim_{n\to\infty}\int_{a}^{b}f_n = \int^{b}_{a}f$ if $\{f_n\}$ converges to $f$ uniformly. See the first part of the exercise for a proof.
	 \[
	|\lim_{n\to\infty}\int_{0}^{1}\frac{g(x)}{n}dx|\leq\lim_{n\to\infty}\int_{0}^{1}\frac{M}{|n|}dx=M\lim_{n\to\infty}\int_{0}^{1}\frac{1}{n}dx=0
	.\] 
Because its absolute value is sandwiched between 0 and 0 we have $\lim_{n\to\infty}\int_{0}^{1}\frac{g(x)}{n}dx$=0, as sought.
\end{solution}
\pagebreak
\begin{problem}
	Find an example of a sequence of continuous functions on $(0,1)$ that converges pointwise to a continuous function on $(0,1)$, but the convergence is not uniform.
\end{problem}
\begin{solution}
	Consider the sequence of functions defined by $f_n(x)=\frac{1}{x+\frac{1}{n}}$. The sequence converges to $f(x)=\frac{1}{x}$.\\
	The sequence converges pointwise. To see this, we write \[
	|\frac{1}{x+\frac{1}{n}}-\frac{1}{x}|=|\frac{x-x-\frac{1}{n}}{x(x+\frac{1}{n})}|=\frac{1}{n}|\frac{1}{x(x+\frac{1}{n})}|
	.\] 
	By choosing $N>\max(\frac{2x^{2}}{\epsilon},\frac{1}{x})$, we can further estimate this:
	\[
	<\frac{1}{n}|\frac{1}{x(x+x)}|=\frac{1}{n}|\frac{1}{2x^2}|<\epsilon
	.\] 
	Hence, the function converges pointwise. However as we have seen in class, it does not converge uniformly.
\end{solution}
\pagebreak
\begin{problem}
	Suppose $f_n:[a,b]\to\mathbb{R}$ is a sequence of continuous functions that converges pointwise to a continuous $f:[a,b]\to\mathbb{R}$. Suppose that for every $x\in[a,b]$, the sequence  $\{|f_n(x)-f(x)|\}$ is monotone. Show that the sequence $\{f_n\}$ converges uniformly.
\end{problem}
\begin{solution}
	Be given $\epsilon>0$, and consider the function $g:x\in[a,b]\to\mathbb{N}$ $|f_n(x)-f(x)|<\epsilon$ is satisfied for all $n\geq N$. We are guaranteed such an  $x$ by the definition of pointwise convergence.\\
	Consider now the function $|f_N(x)-f(x)|$ over the closed bounded interval $x\in[a,b]$. It is continuous, because $f_N, f$ are. By the Extreme Value Theorem, it must attain its maximum on some $x_0\in[a,b]$. If $|f_N(x_0)-f(x_0)|\geq\epsilon$, we can repeat step 1 of this procedure with $x=x_0$ to get an  $N_2$ such that for all  $n>N_2$,  $|f_{N_2}(x_0)-f(x_0)|<\epsilon$.
	Because the sequence $\{|f_n(x)-f(x)|\}$ is monotonically decreasing for all $x$
\end{solution}
\pagebreak
\begin{problem}
	Find a sequence of Lipschitz continuous functions on $[0,1]$ whose uniform limit is  $\sqrt{x}$, which is a non-Lipschitz function.
\end{problem}
\begin{solution}
	Consider the sequence given by
	\[
	f_n(x) = \sqrt{x+\frac{1}{n}}
.\]
Since $\sqrt{x}$ fails Lipschitz continuity when its argument is less than or equal to 0, all elements of the sequence are Lipschitz continuous everywhere on $[0,1]$. In the limit however, we get the same problem studied in class where the slope is undefined at 0.
\end{solution}
\end{document}
