\documentclass[11 pt]{article}
\textwidth=6.5in
\textheight=8.5in
\voffset=-54pt
\oddsidemargin=0in
\evensidemargin=0in
\setlength{\parskip}{8pt}
\setlength{\parindent}{0pt}
\newcommand{\tc}{\textcolor{blue}}
\usepackage[normalem]{ulem}
\usepackage[bottom]{footmisc}
\usepackage{tikz-cd}
\usepackage{amsmath, amssymb, amsthm}
\usepackage{ifthen}
\usepackage[inline]{enumitem}
\setlist{topsep=0pt}
\usepackage{xcolor}
\usepackage{parskip}
\usepackage{blindtext}
\usepackage{multicol}
%operator names
\DeclareMathOperator{\coker}{coker}
\DeclareMathOperator{\im}{im}
\DeclareMathOperator{\coord}{coord}
\DeclareMathOperator{\id}{id}
\DeclareMathOperator{\ob}{\mathrm{ob}}
\DeclareMathOperator{\mor}{\mathrm{mor}}
\DeclareMathOperator*{\colim}{\mathrm{colim}}
\newcommand{\op}{\mathrm{op}}
\newcommand{\co}{\mathrm{co}}
\newcommand{\Nat}{\mathrm{Nat}}
\newcommand{\Hom}{\mathrm{Hom}}
\newcommand{\Map}{\mathrm{Map}}
\newcommand{\End}{\mathrm{End}}
\newcommand{\Aut}{\mathrm{Aut}}
\newcommand{\Sym}{\mathrm{Sym}}
\newcommand{\ev}{\mathrm{ev}}
\newcommand{\Stab}{\mathrm{Stab}}
%blackboard letters
\newcommand{\FF}{\mathbb{F}}
\newcommand{\CC}{\mathbb{C}}
\newcommand{\QQ}{\mathbb{Q}}
\newcommand{\RR}{\mathbb{R}}
\newcommand{\ZZ}{\mathbb{Z}}
\newcommand{\NN}{\mathbb{N}}
\newcommand{\kk}{\mathbbe{k}}
%categories
\newcommand{\A}{\mathcal{A}}
\newcommand{\B}{\mathcal{B}}
\newcommand{\C}{\mathcal{C}}
\newcommand{\D}{\mathcal{D}}
\newcommand{\set}{\mathsf{Set}}
\newcommand{\grp}{\mathsf{Grp}}
\newcommand{\ab}{\mathsf{Ab}}
%theorems, defns, etc
\newtheorem{thm}{Theorem}[section]
\theoremstyle{definition}
\newtheorem*{defn*}{Definition}
\newtheorem*{thm*}{Theorem}
%This makes the problem environment work.
\theoremstyle{definition}
\newtheorem{problem}{Problem}
%This makes the solution environment work.
\makeatletter\newenvironment{solution}[1][Solution]{\par\pushQED{\qed}%
\normalfont 
\topsep6\p@\@plus6\p@\relax\trivlist\item\relax{\itshape#1\@addpunct{.}}\hspace\labelsep\ignorespaces}{%
\popQED\endtrivlist\@endpefalse}
\makeatother

\usepackage{quiver}

\begin{document}
    Name: Lj Gonzales

    Assignment: HW 1 

    Due Date: Friday September 29
    \break
\begin{problem}
(Writing) Continue talking to that engineering friend of yours, one who knows
the material in all of the calculus courses and differential equations, and who knows
you are in this analysis class. In a few short sentences, describe to them the following
concepts. Be very careful to only use terms and ideas that they can understand, but
also be as precise as you possibly can.\\
(1) The uncountability of the real numbers,\\
(2) what it means that a sequence, as a function from N to R, has a limit.
\end{problem}
\begin{solution}
\begin{itemize}
	\item Recall that we called a set countably infinite if we could define a bijective function between it and the natural numbers.
	\item In calculus II we have the idea of a limit to mean what something approaches \emph{as it goes to} something. For example, the limit of
\end{itemize}
\end{solution} 
\pagebreak
\begin{problem}
Writing) Answer the Week 4 5-Minute in-Lecture Drill.
\end{problem}
\begin{solution}
\end{solution}
\pagebreak
\begin{problem}
Show the following sequences converge and find their limits:\\
\begin{center}
(a) $\Bigl\{\frac{(-1)^n}{2n}\Bigr\}$ (b) $\Bigl\{\frac{2^n}{n!}\Bigr\}$ (c) $\Bigl\{\frac{n-cosn}{n}\Bigr\}$ (Hint: Squeeze it.)
\end{center}
\end{problem}
\begin{solution}
\end{solution}
\pagebreak
\begin{problem}
Let $\{x_n\}$be a convergent monotone sequence. Suppose there exists a $k\in\mathcal{N}$ such that $\lim_{n\to\infty}x_n=x_k$. Show that $x_n=x_k$ for all $n\geq k$.
\end{problem}
\begin{solution}
\end{solution}
\pagebreak
\begin{problem}
Let $\{x_n\}$ be a sequence. Suppose there are two convergent subsequences $\{x_n_i\}$ and $\{x_m_i\}$, where\\
\begin{center}
	$\lim_{i\to\infty}x_n_i=a$ and $\lim_{i\to\infty}x_m_i=b,$
\end{center}
and $a\neq b$. Prove that $\{x_n\}$ is not convergent, without using Proposition 2.1.17 in the text.
\end{problem}
\begin{solution}
\end{solution}
\pagebreak
\begin{problem}
	True or false, prove or find a counterexample. If $\{x_n\}$ is a sequence such that $\{x_n^2\}$ converges, then $\{x_n\}$ converges.
\end{problem}
\begin{solution}
	Consider the sequence $\{x_n\}=\{(-1)^n\}$, which we know is not convergent.\\
	Note that the sequence $\{x_n^2\}$ is then the constant sequence $\{1\}$, which is trivially convergent. This, however, did not imply that $\{x_n\}$ is convergent.
\end{solution}
\pagebreak
\begin{problem}
	Do the following:
	\begin{itemize}
	\item Suppose $\{a_n\}$ is a bounded sequence and $\{b_n\}$ is a sequence converging to 0. Show that $\{a_nb_n\}$ converges to 0.
	\item Find an example where $\{a_n\}$ is unbounded, $\{b_n\}$ converges to 0, and $\{a_nb_n\}$ is not convergent.
	\end{itemize}
\end{problem}
\begin{solution}
	\begin{itemize}
		\item We are given that $|a|<M$ for some positive M.
		We then have $|a||b|=|ab|\leq M|b|$ for all  $b\in\{b_n\}$, since $|b|$ is itself a nonnegative number.\\
		On the other hand, we are guaranteed that for every $\frac{\epsilon}{M}$ there exists a $N$ such that for all $n>N$, $|b_n-0|=|b_n|<\frac{\epsilon}{M}$.\\
		Putting these two pieces of information together, we have that $|ab|\leq M|b|<\epsilon$.
		We have thus shown that for any  $\epsilon_1 >0$ (in particular, $\epsilon/M$), there exists a $N$ (the very same) for which $n>N\implies|a_nb_n-0|<\epsilon$.
		By definition, the sequence $\{a_nb_n\}$ must converge to 0
		\item Consider the sequences $\{a_n\}=\{n^2\}$, $\{b_n\}=\{\frac{1}{n}\}$, $\{a_nb_n\}=\{n\}$. This set is exactly the natural numbers, which are unbounded.
	\end{itemize}
\end{solution}
\end{document}
