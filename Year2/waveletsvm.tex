\begin{document}
	Vanishing moments\\
Suppose we have a polynomail that we want to be fully represented within the thumbnail.
	If we have K vanshing moments, then we can do this with a polnomial of degree at most k-1.
	What is the bare minimum in our design that we can get away wth? Suppose we want to trade off the degree k-1.\\
	We want at least one vanishing moment at $k=1$. Then a polynomial of degree 0  (a constant) will be passed fully (DC).
	$\Phi(w)=\frac{1}{\sqrt{2}}\Phi(\frac{w}{2})$ in Fourier $\phi(t)=\sqrt{2}\sum h_0[n]\phi(2t-n)$ \\
	$\Psi(w)=\frac{1}{\sqrt{2}}H_1(w)\Phi(\frac{w}{2})$ in Fourier $\psi$
	We want dot product of shifted phi's to be 0 in time domain. 
	We don't assume anything about perfect reconstruction, perfect filter bank.
	Knowing that the dot product of the phi and psi's are correct, we want to find the conditions that need to be applied on $h_0, h_1$.
	We predict that some of the necessary conditions are going to be orthogonality and at least one vanishing moment.\\
	Want to prove that there is at least one vanishing moment. i.e,  $k\geq 1$. What is our definition of vanishing moment? Recall that it does not explicitly involve $H_1$.
	 $H_0(z)=(1=z^{-1})T(z)$ for some $T(z)$.
	Proof: we want \[
		<\sqrt{2}\sum h_0[k]\phi(2t-k), \sqrt{2}\sum h_0[m]\phi(2t-2n-m)>=\delta[n]
	.\] 
	By Cauchy's formula (?), we have \[
		=2\sum_{k}\sum_{n}h_0[k]h_0[m]<\phi(2t-k), \phi(2t-2n-m)>
		=2\sum_{k}\sum_{m}h_0[k]h_0[m]\frac{1}{2}\delta[k-2n-m]
		=\sum_{k}\sum_{m}h_0[m]h_0[m+2n]=\delta[n]
	.\] 
	Because each of the terms inside of the phi's are integers, we can apply our desired identity on them.\\
	What about \[
	<\phi(2t), \phi(2t)>
	.\]  have to apply a change of variables on its definition \[
	\int_{-\infty}^{\infty}|\phi(2t)|^2dt=\frac{1}{2}
	.\] 
It turns out that if we don't start with an othogonal filter bank, all bets are off. However this proof did not rely on the assumption that we have vanishing moments.
We have \[
	\sum_{m}h_0[n]h_0[m+2n]=\delta[n]
.\] from previously. We can sum over n
\[
	\sum_{n}sum_{m}h_0[n]h_0[m+2n]=\sum_{n}\delta[n]=1
.\] 
We define $K_e=\sum_{m}h_0[2m]$ i.e the sum of all even index coefficients.
Likewise define $K_o=\sum_{m}h_0[2m+1$]$\\
Now we use the fact at at least one vanishing moment means that \[
	H_0(-1)=\sum_{m}h_0[m](-1)^{-m}=0
.\] 
This means that $K_e-K_o=0$ therefore  $K_e=K_o$.
Notice that  \[
	h_0[m]=K_e+K_o=
.\] 
(Recall we had a test question where this was equal to $\sqrt(2)$ but this assumed that we already had a vanishing moment. SO this isn't it)\\
But now we have our definition of $\Phi(w)$ to help us instead. At $w=0$ from our previous definition,
 \[
	 \Phi(0)=\frac{1}{\sqrt{2}}H_0(0)\Phi(0)
.\] Thus $H(0)=\sqrt{2}$ \\
So in the end we have $\sum_{m}h_0[m]=K_e+K_o=\sqrt{2}$, and we need another equation (why? I don't know tbh).\\
\[
	\sum_{m}\sum_{n}h_0[m]h_0[m+2n]\\
	=\sum_{m}\big(\sum_{n}h_0[2m]h_0[2m+2n]+h_0[2m+1]h_0[2m+1+2n]\big)\\

.\] 
We can take out $h_0[2m]$ because it doesn't depend on  $n$.
Notice that in  $\sum_{n}h_0[2m+2n]=K_e$ because it doesn't matter since we're starting from an even index and summing over all other even indexes.\\
\[
	=K_e\sum_{m}h_0[2m]+K_o\sum_{m}h_0[2m+1]
	=K_e^2+K_o^2=1
.\] 
Our given is that $K_e+K_o=\sqrt{2}$.
We can solve this system of equations and arrive at the result that $K_e=K_o$, which is what we ought to prove.\\
Based on our earlier argument, we have thus proved that the assumption of orthogonal decomposition  (the 3 dot product formulas) imply that we need an orthogonal filter bank with $h_0$ and  $h_1$, and that we need at least one vanishing moment.\\
Next time we will go back to the discrete domain because it is more practical for real applications.
\end{document}
